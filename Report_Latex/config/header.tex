%contains all the imports and configurations 

\documentclass[10pt,a4paper]{report}
\usepackage[a4paper, left=2cm, right=2cm, top=2cm, bottom=2cm]{geometry}
\usepackage[utf8]{inputenc}		%utf8 allows also for umlauts
\usepackage[english]{babel}
\usepackage{amsmath}

%Declare new math operators with amsmath:

%atan2 function
\DeclareMathOperator{\atantwo}{atan2}
\DeclareMathOperator{\arctantwo}{arctan2}
\DeclareMathOperator{\acos}{acos}

%Decalre new math font:
\usepackage{mathrsfs}


\usepackage{amsfonts}
\usepackage{amssymb}
\usepackage{graphicx}	%for importing images
\graphicspath{ {./images/} {../GeneratedPlots/} }
\usepackage{gensymb}
\usepackage{siunitx}
%\usepackage[numbered,framed]{matlab-prettifier}
\usepackage{float}
\usepackage[usenames,dvipsnames]{color} % color text
\definecolor{mygreen}{RGB}{28,172,0} % color values Red, Green, Blue
\definecolor{mylilas}{RGB}{170,55,241}

%coloured text
\usepackage{xcolor} %\textcolor{blue}{This is a sample text in blue.}

\usepackage{listings}
\usepackage{alphalph}

\usepackage{breqn} %automatic linebreaking math environment
%insert after all math stuff e.g. amsmath, amssymb mathpazo... is imported

%insert external pdf pages in document
\usepackage{pdfpages} 
% http://texdoc.net/texmf-dist/doc/latex/pdfpages/pdfpages.pdf

%Bibliography settings
\usepackage[citestyle = numeric, bibstyle=numeric]{biblatex} %style =apa %citestyle = numeric, style=numeric, bibstyle = numeric
\addbibresource{References.bib}    

% for including current date into document
\usepackage[ddmmyyyy]{datetime}

%Give captions to items that normally don't have any
\usepackage{caption} 
%use as:
%\DeclareCaptionType{exam}[Example][List of Examples]
%\listofexams
%\begin{enumerate}
%	\item First item.
%	\item Second item.
%	\item Third item.
%	\captionof{exam}{This is a very important list.}
%\end{enumerate}

%For making an appendix
\usepackage[toc,page]{appendix}

% \usepackage[table,xcdraw]{xcolor}

%line drawing in array or tabular
%\usepackage{arydshln} %here used for mimicing matrices in array with dashed lines
%Doesn't work as intended

%Matrix options for dotted lines, etc. etc...
\usepackage{nicematrix} % here used for makin g arrays with dashed separator lines
%http://ctan.math.washington.edu/tex-archive/macros/latex/contrib/nicematrix/nicematrix.pdf

\usepackage{enumitem}
\usepackage{multicol}%try it out!





%acronym package (needs an acronym chapter)
%\usepackage{acronym}
%\usepackage{acro}	%Alternative, more elaborate package for acronyms
%\acsetup{hyperref = true, only-used = true, sort = true,  }

\DeclareAcronym{HAN}{short = HAN , long = Hogeschool van Arnhem en Nijmegen }
\DeclareAcronym{SPC}{short = SPC , long = Smart Production Cell }
\DeclareAcronym{PC}{short = PC , long = Personal Computer }
\DeclareAcronym{DOF}{short DOF = , long = Degree of Freedom }






%\begin{acronym}[Bash]

%	\acro{HAN}{Hogeschool van Arnhem en Nijmegen}
%	\acro{SPC}{Smart Production Cell}
%	\acro{DOF}{degrees of freedom}
%	\acro{FRC}{Fibre Reinforced Composites}
%	\acro{DH}{Denavit-Hartenberg}
%	\acro{IOT}{Internet Of Things}
%	\acro{FHEM}{Freundliche Hausautomation und Energie-Messung \cite{FHEM}}
%	\acro{NFC}{Near-field communication}
%	\acro{IPKW}{Industrial Park Kleevse Waard}
%	\acro{NFC}{Near Field Communication}
%	\acro{ROS}{Robot Operating System}
%	\acro{GUI}{Graphical User Interface}
%	\acro{EOAT}{End Of Arm Tooling}
%	\acro{PL}{Production Line}
%	\acro{ETCS}{European Train Control System}
%	\acro{DT}{Digital Twin}
%	\acro{eq}{equation}
%	\acro{DCS}{Dual Check Safety}
%	\acro{DDS}{Data Distribution System}
%	\acro{fig}{figure}
%	\acro{sect}{section}
%	\acro{OS}{Operating System}
%\end{acronym}


%import PDF pages
\usepackage{pdfpages}
%To include all the pages in the PDF file:
%\includepdf[pages=-]{myfile.pdf}
%To include just the first page of a PDF:
%\includepdf[pages={1}]{myfile.pdf}

%Start new chapter on same page to avaoid big spaces
\usepackage{etoolbox}
\makeatletter
\patchcmd{\chapter}{\if@openright\cleardoublepage\else\clearpage\fi}{}{}{}
\makeatother

% Incluse subsubsection in Toc
\setcounter{tocdepth}{2}
\setcounter{secnumdepth}{3}

 %Wildcard for Signature line
\newcommand*\wildcard[2][6cm]{\vspace*{0.5cm}\parbox{#1}{\hrulefill\par#2}}

%A truly random package
\usepackage{lipsum} %random text

%If custom headers are needed
%\usepackage{fancyhdr}
%
%\pagestyle{fancy}
%\fancyhf{}
%\rhead{Overleaf}
%\lhead{Guides and tutorials}
%\rfoot{Page \thepage}

%for left, center or right aligning figures
\usepackage[export]{adjustbox}
%https://tex.stackexchange.com/questions/91566/syntax-similar-to-centering-for-right-and-left

%Remove word "chapter" but leave numbering
\makeatletter
\renewcommand{\@makechapterhead}[1]{%
	\vspace*{50 pt}%
	{\setlength{\parindent}{0pt} \raggedright \normalfont
		\bfseries\Huge\thechapter.\ #1
		\par\nobreak\vspace{40 pt}}}
\makeatother

%Add line below chapter title and minimze space when text starts
%\usepackage[]{titlesec}
%\titleformat{\chapter}[hang]{\huge\sffamily\bfseries}{}{0pt}{}[\hrule\vspace*{-23pt}] 

%no indentation for paragraphs
\setlength{\parindent}{0cm}

\usepackage{hyperref} %Load last as it can cause trouble with other packages. For Hyperlink https://de.wikibooks.org/wiki/LaTeX-W%C3%B6rterbuch:_hyperref
\hypersetup{
	colorlinks=true,
	linkcolor=blue,
	filecolor=magenta,      
	urlcolor=cyan,
}
\urlstyle{same} %The default is equivalent to \urlstyle{tt}; with \urlstyle{rm} and \urlstyle{sf} the font will be the roman or sans serif upright font. With \urlstyle{same} the current font will be used.

\usepackage{cleveref} %multi reference with \cref{x,y}
% needs to be loaded after hyperref 

%Define \fulref command for nice cross referencing in document with hyperlinks:
\newcommand*{\fullref}[1]{\hyperref[{#1}]{\autoref*{#1} \nameref*{#1}}} 
%How to use: see
% https://tex.stackexchange.com/questions/121865/nameref-how-to-display-section-name-and-its-number

%Making a Glossary
%\usepackage{glossaries}
\usepackage[acronyms,shortcuts,symbols,xindy,nonumberlist]{glossaries-extra}%,numberedsection
%\usepackage[style=long,toc,automake]{glossaries}
%https://en.wikibooks.org/wiki/LaTeX/Glossary
%http://tug.ctan.org/macros/latex/contrib/glossaries/glossariesbegin.pdf
\setabbreviationstyle[acronym]{long-short}

	\newacronym{DCS}{DCS}{Dual Check Safety}
	\newacronym{DDS}{DDS}{Data Distribution System}
	\newacronym{DH}{DH}{Denavit-Hartenberg}
	\newacronym{DOF}{DOF}{degrees of freedom}
	\newacronym{DT}{DT}{Digital Twin}
	%E
	\newacronym{EOAT}{EOAT}{End Of Arm Tooling}
	\newacronym{eq}{eq}{equation}
	\newacronym{ETCS}{ETCS}{European Train Control System}
	%F
	\newacronym{FHEM}{FHEM}{Freundliche Hausautomation und Energie-Messung \cite{FHEM}}
	\newacronym{fig}{fig}{figure}
	\newacronym{FRC}{FRC}{Fibre Reinforced Composites}
	%G
	\newacronym{GUI}{GUI}{Graphical User Interface}
	%H
	\newacronym{HAN}{HAN}{Hogeschool van Arnhem en Nijmegen}
	%I
	\newacronym{IOT}{IOT}{Internet Of Things}
	\newacronym{IPKW}{IPKW}{Industrial Park Kleevse Waard}
	%J
	%K
	%L
	%M
	%N
	\newacronym{NFC}{NFC}{Near-field communication}
	%O
	\newacronym{OS}{OS}{Operating System}
	%P
	\newacronym{PC}{PC}{Personal Computer}
	\newacronym{PL}{PL}{Production Line}
	%Q
	%R
	\newacronym{ROS}{ROS}{Robot Operating System}
	%S
	\newacronym{sect}{sect}{section}
	\newacronym{SPC}{SPC}{Smart Production Cell}
%Glossary entries listed here


\newglossaryentry{numSol}{
	name={Numerical Solution},
	description={Originally the term numerical solution describes a solution given in terms of numbers instead of an explicit closed form expression. In this work, the term "numerical solution" refers to numerical, iterative methods for solving the inverse kinematics problem by using a sequence of steps leading to incrementally better solutions for the joint angles. Examples for these methdods are the jacobian inversion method, the optimization based method cyclic coordinate descent as shown by Lukas Barinka \cite{InvkinMeth_Lukas}. 
}}

\newglossaryentry{forwKin}{name={forward kinematics},description={Forward kinematics describes the position of the endpoint of a kinematic chain in the operational space by the kinematic equations with the joint variables as input. The non-linear kinematic equations map the joint parameters to the configuration of the robot system. This results in a pure geometrical description of motion by means of position, orientation, and their time derivatives.}}

\newglossaryentry{joints}{name={joint},description={A joint is a connection between two links that allows for movement within certain contstraints. }}

\newglossaryentry{link}{name={link},description={A link is a rigid body, defining the
		spatial relationship between two following axes. Links can be rotated or translated by joints, but don't deform themselves.}}
%\glsxtrnewsymbol[description={position}]{x}{\ensuremath{x}}
%\glsxtrnewsymbol[description={velocity}]{v}{\ensuremath{v}}
%\glsxtrnewsymbol[description={acceleration}]{a}{\ensuremath{a}}
%\glsxtrnewsymbol[description={time}]{t}{\ensuremath{t}}
%\glsxtrnewsymbol[description={force}]{F}{\ensuremath{F}}

%\glsxtrnewsymbol[description={}]{}{\ensuremath{}}

\glsxtrnewsymbol[description={joint iteration variable}]{n}{\ensuremath{n}}
\glsxtrnewsymbol[description={link iteration variable}]{i}{\ensuremath{i}}
\glsxtrnewsymbol[description={Number of joints}]{N}{\ensuremath{N}}
\glsxtrnewsymbol[description={Number of links}]{I}{\ensuremath{I}}

\glsxtrnewsymbol[description={local coordinate frame x -axis}]{x_i}{\ensuremath{x_i}}
\glsxtrnewsymbol[description={local coordinate frame y -axis}]{y_i}{\ensuremath{y_i}}
\glsxtrnewsymbol[description={local coordinate frame z -axis}]{z_i}{\ensuremath{z_i}}

\glsxtrnewsymbol[description={rotation around the $x_i$-axis}]{alpha_i}{\ensuremath{\alpha_i}}
\glsxtrnewsymbol[description={translation along $x_i$-axis}]{a_i}{\ensuremath{a_i}}
\glsxtrnewsymbol[description={translation along $z_i$-axis}]{d_i}{\ensuremath{d_i}}
\glsxtrnewsymbol[description=rotation around $z_i$-axis{}]{theta_i}{\ensuremath{\theta_i}}

\glsxtrnewsymbol[description={Vector}]{v}{\ensuremath{v}}

\glsxtrnewsymbol[description={normal vector}]{n_xyz}{\ensuremath{n_{xyz}}}
\glsxtrnewsymbol[description={orientation vector}]{o_xyz}{\ensuremath{o_{xyz}}}
\glsxtrnewsymbol[description={approach vector}]{a_xyz}{\ensuremath{a_{xyz}}}
\glsxtrnewsymbol[description={position vector}]{p_xyz}{\ensuremath{p_{xyz}}}

\glsxtrnewsymbol[description={x-position within frame}]{x}{\ensuremath{x}}
\glsxtrnewsymbol[description={y-position within frame}]{y}{\ensuremath{y}}
\glsxtrnewsymbol[description={z-position within frame}]{z}{\ensuremath{z}}
\glsxtrnewsymbol[description={rotation around x}]{alpha}{\ensuremath{\alpha}}
\glsxtrnewsymbol[description={rotation around y}]{beta}{\ensuremath{\beta}}
\glsxtrnewsymbol[description={rotation around z}]{gamma}{\ensuremath{\gamma}}

\glsxtrnewsymbol[description={Transformation matrix}]{T}{\ensuremath{T}}
\glsxtrnewsymbol[description={Rotation matrix}]{R}{\ensuremath{R}}

\glsxtrnewsymbol[description={$\sin(\theta_i)$}]{s_i}{\ensuremath{s_i}}
\glsxtrnewsymbol[description={$\cos(\theta_i)$}]{c_i}{\ensuremath{c_i}}

\glsxtrnewsymbol[description={Position with orientation in euclidean space $p=(x,y,z,\alpha,\beta,\gamma)$}]{p}{\ensuremath{p}}
\glsxtrnewsymbol[description={joint angles in joinet space $q=(q_1, q_2, q_3, q_4, q_5, q_6)$ }]{q}{\ensuremath{q}}

\glsxtrnewsymbol[description={Jacobian matrix}]{J}{\ensuremath{J}}
\glsxtrnewsymbol[description={Joint-space inertia matrix}]{M}{\ensuremath{M}}
\glsxtrnewsymbol[description={Coriolis and centripetal coupling matrix}]{C}{\ensuremath{C}}
\glsxtrnewsymbol[description={force}]{F}{\ensuremath{F}}
\glsxtrnewsymbol[description={Gravity loading}]{G}{\ensuremath{G}}
\glsxtrnewsymbol[description={Wrench applied at the end effector}]{W}{\ensuremath{W}}

\glsxtrnewsymbol[description={generalized actuator forces}]{Q}{\ensuremath{Q}}
\glsxtrnewsymbol[description={actuator torque}]{Upsilon}{\ensuremath{\Upsilon}}

\glsxtrnewsymbol[description={Lagrangian}]{L}{\ensuremath{L}}
\glsxtrnewsymbol[description={Kinetic Energy}]{KinEner}{\ensuremath{T(q,\dot{q})}}
\glsxtrnewsymbol[description={Potential energy}]{PotEner}{\ensuremath{V(q)}}

\glsxtrnewsymbol[description={Innertia Matrix}]{InertiaMatr}{\ensuremath{\mathcal{I}}}

\glsxtrnewsymbol[description={link mass}]{m}{\ensuremath{m}}
\glsxtrnewsymbol[description={width}]{w}{\ensuremath{w}}
\glsxtrnewsymbol[description={height}]{h}{\ensuremath{h}}
\glsxtrnewsymbol[description={length}]{l}{\ensuremath{k}}
\glsxtrnewsymbol[description={Christoffel symbol}]{Christoffel}{\ensuremath{\Gamma_{...}}}


%\glsxtrnewsymbol[description={}]{}{\ensuremath{}}

%Note: Don't use dots in filenames, as this can lead to errors, e.g. with glossary package. Here the filename is cut after the dot, so wrong filenames are used for processing and will lead to errors.

%remove clear page before printglossary
\renewcommand*{\glsclearpage}{}



\makeglossaries
%\makenoidxglossaries



%Include list of figures and list of tables in TOC
\usepackage{tocbibind}


%Set some things clear 
\newcommand{\HANSupervisor}{Wesselingh Ellen}
\newcommand{\CompanySupervisor}{Nguyen Trung}
\newcommand{\AdditionalSupervisor}{Jeltsema Dimitri}
\author{Karl Wallkum}

%% Preamble 
%Deactivate before Distribution!!

