\section{Denavit-Hartenberg-Convention}

The\ac{DH}-Convention is a commonly used and simplifies the forward and backward transformation. It was named after Jaques Denavit and Richard Hartenberg who developed a general theory to describe a serial link mechanism. \cite{DenavitHartenbergLesson}\\
It consists of following parts:

\begin{itemize}[leftmargin=3cm]
	\item \ac{DH}-Convention for establishing the coordinate systems
	\item \ac{DH}-Transformation for generation of the coordinate systems
	\item \ac{DH}-Parameters as a result form the transformations
\end{itemize}

Determining the coordinate systems is done according to set rules. Nevertheless, the choices of coordinate frames are also not uniqe, so different peole will derive different, but correct frame assignments. This freedom of choice should be used to bring as many \ac{DH}-Parameters as possible to zero. This simplifies subsequent equations and calculations. \cite{DenavitHartenbergKonventionen}

Each joint of the robot is described by four parameters.