\section{Jacobian} \label{sec:Jacobian}
When the position of the end-effector is $\gls{p}=(x,y,z,\alpha,\beta,\gamma)$ and the joint angles are  $\gls{q}=(q_1, q_2, q_3, q_4, q_5, q_6)$, the derivative of $p$ with respect to the joint variables $q$ gives the Jacobian matrix as seen in  \cref{eq:JacDeriv}.\\
\begin{equation} \label{eq:JacDeriv}
	\frac{dp}{dq}=J(q)
\end{equation}
\medskip
As a result, the Jacobian connects the velocities of the two state-spaces as seen in equation \ref{eq:JacVel} ( see \cite{CorkeRoboticVisionControl} p.231, sec. 8.1 ). Further derivations like acceleration, Jerk, Snap an Crackle can be obtained similarly by dividing both sides through $dt$. \\
\begin{equation}\label{eq:JacVel}
	\dot{p}=J(q)\dot{p}
\end{equation}
As the Jacobian matrix $\gls{J}$ connects the end-effector space in generalized coordinates with the joint space in angle coordinates, it can relate also forces between the state-spaces according to the principle of virtual work (see \cite{IndustrialRobotArm} page 157, section 5.11 ).\\
These are a few notable conversions, the Jacobian provides:\\
$ \dot{x}=J(\theta)\dot{\theta} $ \\
$ \theta=J{^-1}x $ \\
$ \dot{\theta} = J^{-1} \dot{x} $\\
$ \ddot{\theta} = J^{-1} \ddot{x} + \frac{d}{dt}(J^{-1}) \dot{x} $\\
