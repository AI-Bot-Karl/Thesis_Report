\section{Dynamics}

With the kinematic model in place, dynamics can be attached. The motion of the \ac{EOAT} is a result of the motion of each link, which are moved by torques applied by the \glspl{joints}. These torques have to counteract friction, gravity and varying inertias.
With the rigid-body equations of motion, the robot is described by a set of coupled dynamic equations. These give the joint torques necessary to move to and hold a certain pose.

%\subsection{Rigid-Body equations of motion}
\subsection{Inverse Dynamics}

For a serial link manipulator, the equations of motion can be expressed as a set of coupled differential equations in matrix form as seen in equation \ref{eq:RBeqMot}. These equations describing the manipulator rigid-body dynamics are called inverse dynamics.
The inverse dynamics is described by following terms:
\begin{itemize}
	\item[$ M $] Joint-space inertia matrix
	\item[$ C $] Coriolis and centripetal coupling matrix
	\item[$ F $] Friction force
	\item[$ G $] Gravity loading
	\item[$ W $] Wrench applied at the end effector
	\item[$ J $] Jacobian of the robot
\end{itemize}
The equations are given in generalized joint coordinates $\gls{q}$, velocities $\dot{q}$ and accelerations $\ddot{q}$.
The result is the vector $Q$ of generalized actuator forces associated with the generalized coordinates $ q$, that can be used to control the robot.

\begin{equation}\label{eq:RBeqMot}
	Q=M(q)\ddot{q}+C(q,\dot{q})\dot{q}+F(\dot{q})+G(q)+J(q)^TW
\end{equation}


\subsection{Derivation of the Equations of motion}
The equations of motion can be derived with a Lagrangian approach as seen in \cite{MathIntroRobManip}, chapter 3.

\subsubsection{Lagrange equations for an open-chain manipulator}

The equations of motion for an open-chain robot can be formed by substituting the Lagrangian into the Lagrange equations. 
The Lagrange equations in vector form can be seen in eq. \ref{eq:LagrEq}. They are given for a mechanical system with generalized coordinates $\gls{q}$ and with $\gls{Upsilon}$ representing the actuator torque.
\begin{equation} \label{eq:LagrEq}
\frac{d}{dt}\frac{\partial L}{\partial \dot{q}} - \frac{\partial L}{\partial q} = \Upsilon
\end{equation}

With the Lagrangian in place, expanding the terms of partial derivatives and rearranging, these terms form eq. \ref{eq:RBeqMot} as seen in \cite{MathIntroRobManip}, section 3.2.\\

\paragraph{Lagrangian for an open-chain robot}

The Lagrangian \gls{L} can be calculated with the potential \gls{PotEner} and kinetic \gls{KinEner} energy as seen in eq. \ref{eq:Lagrangian}. 
\begin{equation} \label{eq:Lagrangian}
	L(q,\dot{q}) = T(q,\dot{q}) - V(q)
\end{equation} 



\paragraph{Kinetic Energy of the links}
%The kinetic energy of the ith link can be found in Math. Intro p.168  eq. 4.17
For an open chain robot, the total kinetic energy can be defined as in equation \ref{eq:TotKinEn}.
\begin{equation}\label{eq:TotKinEn}
T(\theta,\dot{\theta}) = \sum_{i=1}^{n} T_i(\theta,\dot{\theta}) = \frac{1}{2} \dot{\theta}^T M(\theta)\dot{\theta}
\end{equation}

\paragraph{Potential Energy of the links}
%The potential energy of the ith link can be found in Math. Intro p.169
The total potential energy is defined similarly as in equation \ref{eq:TotPotEn}
\begin{equation}\label{eq:TotPotEn}
V(\theta)=\sum_{i=1}^{n} V_i(\theta) = m_i*g*h_i(\theta)
\end{equation}

\paragraph{Forming the Lagrangian}
Substituting these into the Lagrangian gives eq. \ref{eq:LagrangianRobManip}.
\begin{equation}\label{eq:LagrangianRobManip}
	L(\theta, \dot{\theta})= \sum_{i=1}^{n} (T_i(\theta,\dot{\theta})- V_i(\theta)) = \frac{1}{2} \dot{\theta}^T M(\theta)\dot{\theta}-V(\theta)
\end{equation}



%\subsubsection{Equations of motion for a 6DOF robot arm}
%
%\subsubsection{Lagrangian for a 6DOF robot arm}

%\paragraph{Potential Energy for an 6DOF robot arm}
%
%as in three link manipulator, page 17
%
%\paragraph{Kinetic Energy for a 6DOF robot arm}



\subsubsection{Inertia Matrix }

"The joint-space inertia is a positive definite, and therefore symmetric, matrix". \cite{CorkeRoboticVisionControl}, section 9.2.2

The manipulator inertia matrix can be calculated with the link inertia Matrix and the Jacobian as seen in equation \ref{eq:ManipInMatr}.
\begin{equation}\label{eq:ManipInMatr}
	M(\theta) = \sum_{i=1}^{n} J_i^T(\theta) M_i J_i(\theta)
\end{equation}

\paragraph{Link inertia Matrix}
The link inertia Matrix can be created as in eq. \ref{eq:LinkInMatrix} with the Inertia Tensor \gls{InertiaMatr}and the link mass \gls{m}.
\begin{equation}\label{eq:LinkInMatrix}
	M= 
	\begin{bmatrix}
	mI & 0\\
	0  & \mathcal{I}
	\end{bmatrix}
\end{equation}

\paragraph{Inertia Tensor}
The inertia Tensor
 $\mathcal{I}$ 
can be approximated with homogeneous simple forms like the cuboid or a cylinder as given in \cite{PhysScientEng}. 
An example for the inertia Tensor can be found in \cite{MathIntroRobManip} 

\paragraph{Inertia Matrix example}
An example for the inertia matrix is given in eq. \ref{eq:InMatrHomBar} for a homogeneous bar with width $\gls{w}$, height $\gls{h}$ and length $\gls{l}$. Other geometric shapes and non-homogeneous structures have different inertia matrices. 
\begin{equation} \label{eq:InMatrHomBar}
	M=
	\begin{bmatrix}
	m & 0 & 0 & 0 & 0 & 0\\
	0 & m & 0 & 0 & 0 & 0\\
	0 & 0 & m & 0 & 0 & 0\\
	0 & 0 & 0 & \frac{m}{12}(w^2 + h^2) & 0 & 0\\
	0 & 0 & 0 & 0 & \frac{m}{12}(l^2+h^2) & 0\\
	0 & 0 & 0 & 0 & 0 & \frac{m}{12}(l^2 +w^2)\\
	\end{bmatrix}
\end{equation}


%Translational motion of a body can be described in the inertial frame by Newton's second law as seen in eq. \ref{eq:NewtSecLaw}. It is simplified as a pointmass $m$ at position $x$. It gives the acceleration of the body due to the forces acting on it.
%\begin{equation} \label{eq:NewtSecLaw}
%	m * \ddot{x} = f
%\end{equation}
%
%With Euler's equations of motion, the rotational motion of a body in SO(3) (3D rotation group) can be described as in eq. \ref{label}. It gives the angular accelerations of the body in the body frame due to the applied torque acting on it.
%\begin{equation}
%	
%\end{equation}


\subsubsection{Coriolis Matrix }
%The Coriolis matrix can be calculated as seen in Math. Intro p.171  eq. 4.23
The Coriolis matrix can be calculated with \cref{eq:CoriolisMatr} (see, \cite{MathIntroRobManip}, sec. 4.3, p. 172)
\begin{equation}\label{eq:CoriolisMatr}
	C_{ij}(\theta,\dot{\theta}) = \sum_{k=1}^{n} \Gamma_{ijk}\dot{\theta}_k=\frac{1}{2} \sum_{k=1}^{n} (\frac{\partial M_{ij}}{\partial\theta_k}+\frac{\partial M_{ik}}{\partial\theta_j}-\frac{\partial M_{kj}}{\partial\theta_i})\dot{\theta}_k
\end{equation}

\subsubsection{Friction }
The friction can be determined through friction models as given in  \cite{CorkeRoboticVisionControl}, section 9.1.2.
As friction models are computationally very intensive, they will not be subject to this simulation of a robot arm.

\subsubsection{Gravity Term}
As  $V (\theta)$ is the potential energy due to gravity, the forces acting on the joint due to gravity can be determined by $dV/d\theta_i$. %see page 169

\subsubsection{Wrench at Endeffector }
A wrench $W$ is a combination of translational force $(f)$ and rotational moment $(m)$ into a vector with six elements:\\
$ W = (f_x, f_y, f_z, m_x, m_y, m_z)$ (see \cite{CorkeRoboticVisionControl} page 69, section 3.2.2 )\\
This wrench works on the endeffector, so it is given in the Cartesian workspace.
As seen in \fullref{sec:Jacobian}, these forces can be directly transformed into joint space with the Jacobian.









\subsection{forward Dynamics}
To simulate the effects of the joint torques on the robot, the equations of motion (eq. \ref{eq:RBeqMot}) can be rearranged as seen in equation \ref{eq:FWDyn}. This gives the resulting joint accelerations $\ddot{q}$ which can be used to calculate the movements of the links.

\begin{equation} \label{eq:FWDyn}
	\ddot{q}= M^{-1}(q) (Q-C(q,\dot{q})\dot{q} - F(\dot{q}) -G(q) -J(q)^T W)
\end{equation}

%\newpage