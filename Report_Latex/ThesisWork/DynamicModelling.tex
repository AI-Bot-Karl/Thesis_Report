\section{Dynamic Modelling}

With the kinematic model in place, dynamics can be attached. The motion of the \ac{EOAT} is a result of the motion of each link, which are moved by torques applied by the joints. These torques have to counteract friction, gravity and varying inertias.
With the rigid-body equations of motion, the robot is described by a set of coupled dynamic equations. These give the joint torques necessary to move to and hold a certain pose.

%\subsection{Rigid-Body equations of motion}
\subsection{inverse Dynamics}

For a serial link manipulator, the equations of motion can be expressed as a set of coupled differential equations in matrix form as seen in equation \ref{eq:RBeqMot}. These equations describing the manipulator rigid-body dynamics are called inverse dynamics.
The inverse dynamics is described by following terms:
\begin{itemize}
	\item[$ M $] joint-space inertia matrix
	\item[$ C $] coriolis and centripetal coupling matrix
	\item[$ F $] friction Force
	\item[$ G $] gravity loading
	\item[$ W $] Wrench applied at the end effector
	\item[$ J $] Jacobian of the robot
\end{itemize}
The equations are given in generalized joint coordinates $q$, velocities $\dot{q}$ and accelerations $\ddot{q}$.
The result is the vector $Q$ of generalized actuator forces associated with the generalized coordinates $ q$, that can be used to control the robot.

\begin{equation}\label{eq:RBeqMot}
	Q=M(q)\ddot{q}+C(q,\dot{q})\dot{q}+F(\dot{q})+G(q)+J(q)^TW
\end{equation}


\subsection{title}




\subsection{forward Dynamics}
To simulate the effects of the joint torques on the robot, the equations of motion (eq. \ref{eq:RBeqMot}) can be rearranged as seen in equation \ref{eq:FWDyn}. This gives the resulting joint accelerations $\ddot{q}$ which can be used to calculate the movements of the links.

\begin{equation} \label{eq:FWDyn}
	\ddot{q}= M^{-1}(q) (Q-C(q,\dot{q})\dot{q} - F(\dot{q}) -G(q) -J(q)^T W)
\end{equation}