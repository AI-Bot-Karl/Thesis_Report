% !TeX spellcheck = <none>
%testtest
\chapter{Preface}
%Checklist Preface: https://www.scribbr.com/dissertation/preface-dissertation/

%% definition/clickbait
Robots can be defined as programmable movement automatons that can perform tasks without human supervision and can be taught at least repetitive tasks. Increasingly, also ways to sense their surroundings are added and improve their movements according to their surroundings. These additionally to the sensors like pulse encoders at their axes to feedback control their endpoint position accuracy. 
%system supervision - robotic system - 
% A robot is defined as a programmable movement automaton, with the ability to perform tasks without human supervision. Not only can the robot be taught repetitive tasks, that it can execute with pulse encoders at their axes. it can as well sense its surroundings and on the basis of that information improve movements accordingly. 
% go to old from new! Go from general to specific Don't jump from general to specific to examples, don't mix it because reader has then to do all the work. and show that you are not only able to understand concepts, but are able to 1) present them in a coherent way and (later) critique them. chronology
\\

%% simple robots
I have started working with robots and robotic systems in my bachelor studies. As a starting engineer, I was exploring the possibilities of automated manufacturing with CNC mills and 3D printers. These were very simplistic robotic systems based on a feedforward control with stepper motors for position accuracy. For starting a production process, these devices had to be half automatically calibrated and the position and orientation needed to be taught automatically by pointing the drill/printing head to the markerpoints. 

%% current robots


%% future robot (envisioned)

%What are keywords for each of the paragraphs? e.g. HUMAN supervision, robotic Systems, etc..
%Use strong Verbs!
%signalwords
%
%From Corce Robotics Toolbox: (Refactor!)
%The term robot first appeared in a 1920 Czech science fiction play “Rossum’s Universal 
%Robots” by Karel .apek (pronounced Chapek). The term was coined by his brother 
%Josef, and in the Czech language means serf labor but colloquially means hardwork 
%or drudgery. The robots in the play were artificial people or androids and as in so 
%many robot stories that follow this one, the robots rebel and it ends badly for human-
%ity. Isaac Asimov’s robot series, comprising many books and short stories written be-
%tween 1950 and 1985, explored issues of human and robot interaction and morality. 
%The robots in these stories are equipped with “positronic brains” in which the “Three 
%
%
%
%laws of robotics” are encoded. These stories have influenced subsequent books and 
%movies which in turn have shaped the public perception of what robots are. The mid 
%twentieth century also saw the advent of the field of cybernetics  – an uncommon term 
%today but then an exciting science at the frontiers of understanding life and creating 
%intelligent machines.
%The first patent for what we would now consider a robot was filed in 1954 by 
%George C. Devol and issued in 1961. The device comprised a mechanical arm with a gripper that was mounted on a track and the sequence of motions was encod-
%ed as magnetic patterns stored on a rotating drum. The first robotics company, 
%Unimation, was founded by Devol and Joseph Engelberger in 1956 and their first 
%industrial robot shown in Fig. 1.2 was installed in 1961. (Unimation) (Get picture from children book)