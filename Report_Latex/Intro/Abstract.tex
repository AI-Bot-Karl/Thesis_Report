\section{Abstract}

%~150Words, currently 163

%\begin{abstract} %can be done to put Abstract on single page without number
	This work aims to integrate a FANUC 210F 6 axis industrial robot arm  into an experimental production line. 
As this production line is set in a research environment, gaining a deeper understanding of all involved systems is desired.\\
\\
The dynamic behaviour of a physical system is best expressed with an analytical model.
In order to control a robot arm, a kinematic model needs to be created. By attaching inertias and external forces, a dynamic model is obtained. With this model, a control algorithm can be derived.\\
\\
The objective of this thesis is to derive the complete inverse kinematic model of a 6 \ac{DOF} robotic arm analytically. For an exact numerical simulation of the device most steps are laid out theoretically and difficulties in the practical implementation are described.
Additionally for follow up projects this work also contains a quick start guide and a safety manual for the robot in this setting .
Finally to contribute to current research, twinning specifications will be defined.\\

%\end{abstract}

%See: https://www.sfu.ca/~jcnesbit/HowToWriteAbstract.htm
%
%%What is an Abstract?
%
%The abstract is an important component of your thesis. Presented at the beginning of the thesis, it is likely the first substantive description of your work read by an external examiner. You should view it as an opportunity to set accurate expectations.
%The abstract is a summary of the whole thesis. It presents all the major elements of your work in a highly condensed form.
%An abstract often functions, together with the thesis title, as a stand-alone text. Abstracts appear, absent the full text of the thesis, in bibliographic indexes such as PsycInfo. They may also be presented in announcements of the thesis examination. Most readers who encounter your abstract in a bibliographic database or receive an email announcing your research presentation will never retrieve the full text or attend the presentation.
%An abstract is not merely an introduction in the sense of a preface, preamble, or advance organizer that prepares the reader for the thesis. In addition to that function, it must be capable of substituting for the whole thesis when there is insufficient time and space for the full text. 
%
%
%%Size and Structure
%
%Currently, the maximum sizes for abstracts submitted to Canada's National Archive are 150 words (Masters thesis) and 350 words (Doctoral dissertation)
%The structure of the abstract should mirror the structure of the whole thesis, and should represent all its major elements.
%%For example, if your thesis has five chapters (introduction, literature review, methodology, results, conclusion), there should be one or more sentences assigned to summarize each chapter. 
%
%
%%Clearly Specify Your Research Questions
%
%As in the thesis itself, your research questions are critical in ensuring that the abstract is coherent and logically structured. They form the skeleton to which other elements adhere.
%They should be presented near the beginning of the abstract.
%There is only room for one to three questions. If there are more than three major research questions in your thesis, you should consider restructuring them by reducing some to subsidiary status. 
%
%Don't Forget the Results
%
%The most common error in abstracts is failure to present results.
%The primary function of your thesis (and by extension your abstract) is not to tell readers what you did, it is to tell them what you discovered. Other information, such as the account of your research methods, is needed mainly to back the claims you make about your results.
%Approximately the last half of the abstract should be dedicated to summarizing and interpreting your results. 


%%Example and Tips: https://www.scribbr.de/aufbau-und-gliederung/abstract-schreiben/






